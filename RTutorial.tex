\documentclass[12pt]{article}
\usepackage{graphicx,verbatim}
\setlength{\topmargin}{-.8 in}
\setlength{\textheight}{9.4  in}
\setlength{\oddsidemargin}{-.1in}
\setlength{\evensidemargin}{-.1in}
\setlength{\textwidth}{6.35in}
\include{macros}
\include{xtable}
\usepackage{Sweave}
\begin{document}
\input{RTutorial-concordance}


\title{R Tutorial}
\date{September 2, 2015}

\maketitle

\section{Introduction}

\begin{enumerate}

\item Open R or RStudio. 
\item R and RStudio have a console and a script file. 
\item Important R things:
  \begin{itemize}
  \item R is a programming language
  \item R is open-source
  \item Today R has 7096 packages available and that number continues to grow
  \item R is well documented through help files and Google
  \item R is case sensitive
  \end{itemize}
  
\item R can be a calculator
\begin{Schunk}
\begin{Sinput}
> 1 + 2 
\end{Sinput}
\begin{Soutput}
[1] 3
\end{Soutput}
\end{Schunk}

\item We can assign variables
\begin{Schunk}
\begin{Sinput}
> x = 1
> y <- 2
\end{Sinput}
\end{Schunk}

\item We can use built in functions
\begin{Schunk}
\begin{Sinput}
> a <- sum(x,y)
> a
\end{Sinput}
\begin{Soutput}
[1] 3
\end{Soutput}
\begin{Sinput}
> b <- c(x,y)
> b
\end{Sinput}
\begin{Soutput}
[1] 1 2
\end{Soutput}
\end{Schunk}

\item We can comment out lines of code
\begin{Schunk}
\begin{Sinput}
> y <- function(x) sum(x)      # Comment about function here
> # Or comment about function here
\end{Sinput}
\end{Schunk}

\item We can install packages, using the {\tt install.packages()} command.
\itme We can get help using the {\tt help()} command or typing a ? before the function/package in the console.

\end{enumerate}

\section{Let's Do Some Analysis!}
\begin{enumerate}

\item What is your working directory?

\begin{Schunk}
\begin{Sinput}
> getwd()
\end{Sinput}
\begin{Soutput}
[1] "C:/Users/allisontheobold/Documents/R Workshop"
\end{Soutput}
\begin{Sinput}
> 
> # setwd()
> # Use to reset your working directory
\end{Sinput}
\end{Schunk}

\item Read in the data:

\begin{Schunk}
\begin{Sinput}
> #ops_chunk$set(fig.width=5, fig.height=4, out.width='\\linewidth', dev='pdf', concordance=TRUE, size = 'footnotesize')
> 
> options(replace.assign=TRUE,width=72, digits = 3, max.print="72",
+         show.signif.stars = FALSE)
> blackft <- read.csv("C:/Users/allisontheobold/Documents/R Workshop/BlackfootFish.csv")
> 
> # Could also use
> # blackft <- read.csv(
> #      "http://www.math.montana.edu/~jimrc/classes/stat505/data/BlackfootAllFish.csv")
\end{Sinput}
\end{Schunk}

\item Look at the data:
    \begin{itemize}
    \item Look over the dataset
    \item Look over the top of the data set
    \item Look over the summary of the data set
    \end{itemize}
    
\begin{Schunk}
\begin{Sinput}
> blackft
\end{Sinput}
\begin{Soutput}
      trip mark length  weight year     section species
1        1    0  288.0  175.00 1989    Johnsrud     RBT
2        1    0  288.0  190.00 1989    Johnsrud     RBT
3        1    0  285.0  245.00 1989    Johnsrud     RBT
4        1    0  322.0  275.00 1989    Johnsrud     RBT
5        1    0  312.0  300.00 1989    Johnsrud     RBT
6        1    0  363.0  380.00 1989    Johnsrud     RBT
7        1    0  269.0  170.00 1989    Johnsrud     RBT
8        1    0  160.0   40.00 1989    Johnsrud     RBT
9        1    0  213.0   80.00 1989    Johnsrud     RBT
10       1    0  157.0   35.00 1989    Johnsrud     RBT
 [ reached getOption("max.print") -- omitted 18342 rows ]
\end{Soutput}
\begin{Sinput}
> head(blackft)
\end{Sinput}
\begin{Soutput}
  trip mark length weight year  section species
1    1    0    288    175 1989 Johnsrud     RBT
2    1    0    288    190 1989 Johnsrud     RBT
3    1    0    285    245 1989 Johnsrud     RBT
4    1    0    322    275 1989 Johnsrud     RBT
5    1    0    312    300 1989 Johnsrud     RBT
6    1    0    363    380 1989 Johnsrud     RBT
\end{Soutput}
\begin{Sinput}
> summary(blackft)
\end{Sinput}
\begin{Soutput}
      trip          mark           length        weight    
 Min.   :1.0   Min.   :0.000   Min.   : 16   Min.   :   0  
 1st Qu.:1.0   1st Qu.:0.000   1st Qu.:186   1st Qu.:  65  
 Median :2.0   Median :0.000   Median :250   Median : 150  
 Mean   :1.5   Mean   :0.093   Mean   :262   Mean   : 246  
 3rd Qu.:2.0   3rd Qu.:0.000   3rd Qu.:330   3rd Qu.: 330  
 Max.   :2.0   Max.   :1.000   Max.   :986   Max.   :4677  
                                             NA's   :1796  
      year             section       species     
 Min.   :1989   Johnsrud   :11648   Brown: 3171  
 1st Qu.:1991   ScottyBrown: 6704   Bull :  553  
 Median :1996                       RBT  :12341  
 Mean   :1997                       WCT  : 2287  
 3rd Qu.:2002                                    
 Max.   :2006                                    
\end{Soutput}
\end{Schunk}

\item Discover what types of variables the data are.

\begin{Schunk}
\begin{Sinput}
> class(blackft$section)
\end{Sinput}
\begin{Soutput}
[1] "factor"
\end{Soutput}
\begin{Sinput}
> str(blackft$section)
\end{Sinput}
\begin{Soutput}
 Factor w/ 2 levels "Johnsrud","ScottyBrown": 1 1 1 1 1 1 1 1 1 1 ...
\end{Soutput}
\begin{Sinput}
> class(blackft$species)
\end{Sinput}
\begin{Soutput}
[1] "factor"
\end{Soutput}
\begin{Sinput}
> class(blackft$year)
\end{Sinput}
\begin{Soutput}
[1] "integer"
\end{Soutput}
\begin{Sinput}
> str(blackft$year)
\end{Sinput}
\begin{Soutput}
 int [1:18352] 1989 1989 1989 1989 1989 1989 1989 1989 1989 1989 ...
\end{Soutput}
\begin{Sinput}
> str(blackft)
\end{Sinput}
\begin{Soutput}
'data.frame':	18352 obs. of  7 variables:
 $ trip   : int  1 1 1 1 1 1 1 1 1 1 ...
 $ mark   : int  0 0 0 0 0 0 0 0 0 0 ...
 $ length : num  288 288 285 322 312 363 269 160 213 157 ...
 $ weight : num  175 190 245 275 300 380 170 40 80 35 ...
 $ year   : int  1989 1989 1989 1989 1989 1989 1989 1989 1989 1989 ...
 $ section: Factor w/ 2 levels "Johnsrud","ScottyBrown": 1 1 1 1 1 1 1 1 1 1 ...
 $ species: Factor w/ 4 levels "Brown","Bull",..: 3 3 3 3 3 3 3 3 3 3 ...
\end{Soutput}
\end{Schunk}

\item Remove Bull trout and WCT (whitefish) and any fish with missing weight.

\begin{Schunk}
\begin{Sinput}
> blackft.sub <- droplevels(subset(blackft, !is.na(weight) & species != "WCT" & species != "Bull"))
> 
> # droplevels() drops unused levels from a factor
> # subset() takes in the original data set and you insert logical expressions
> #    indicating elements or rows to keep
> 
> # The != indicates not equal to
> # "NA", "Bull" and "WCT" are in quotes because they are strings 
\end{Sinput}
\end{Schunk}

\item What are the different values of species in the new subsetted data? Convert the species of the new data frame into a factor with only 2 levels, Brown and RBT.

\begin{Schunk}
\begin{Sinput}
> str(blackft$species)
\end{Sinput}
\begin{Soutput}
 Factor w/ 4 levels "Brown","Bull",..: 3 3 3 3 3 3 3 3 3 3 ...
\end{Soutput}
\begin{Sinput}
> levels(blackft$species)
\end{Sinput}
\begin{Soutput}
[1] "Brown" "Bull"  "RBT"   "WCT"  
\end{Soutput}
\begin{Sinput}
> blackft.sub$species <- factor(blackft.sub$species, levels = c("Brown","RBT"))
> summary(blackft$species)
\end{Sinput}
\begin{Soutput}
Brown  Bull   RBT   WCT 
 3171   553 12341  2287 
\end{Soutput}
\end{Schunk}

\item Use a plot to show how the proportion of RBT changes with year and section. Does one section always have a higher proportion RBT than the other?

\begin{Schunk}
\begin{Sinput}
> blackft.sub$year <- as.factor(blackft.sub$year)
> # converts year to a factor instead of an integer
> 
> countTable <- with(blackft.sub, tapply(species, blackft.sub[, c(5,7,6)], length))
> # Want to use use the data set to summarize year, species and section
> 
> # tapply() takes in a vector/obect and a list of one or more factors
> #     and applies a function to them
> 
> # Here species is the vector, the index is (year, species, section), and length is the function
> 
> countData <- as.data.frame(countTable)
> # converts countTable to dataframe instead of an array
> 
> propJohn <- countData[,2]/(countData[,1]+countData[,2])
> propScotty <- countData[,4]/(countData[,3]+countData[,4])
> year <- unique(blackft$year)
> plot(year,propJohn,xlab="Year",ylab="Proportion of RBT",col="green", 
+           type = "b", lty=1, ylim=c(0.45, 1))
> points(year, propScotty, col="blue", type="b", lty=2)
> legend('bottomleft', legend = c("Johnsrud", "Scotty Brown"), lty=c(1,2), cex=0.85, merge=TRUE)
\end{Sinput}
\end{Schunk}

\item Cool plot with built in functions
\begin{Schunk}
\begin{Sinput}
> require(ggplot2)
> conditionlow <- function(x)(50*x^{1/3}/1.85)
> conditionhigh <- function(x)(50*x^{1/3}/0.65)
> blackft.sub$species <- factor(blackft.sub$species, levels = c("Brown","RBT"))
> qplot(weight, length, data=blackft.sub, facets=~species) + 
+           stat_function(fun=conditionlow) + stat_function(fun=conditionhigh)  
> 
\end{Sinput}
\end{Schunk}

\item Declaring a new variable

\begin{Schunk}
\begin{Sinput}
> blackft.sub$conditionNumber <- with(blackft.sub, weight^{1/3}/length*50)
> blackft.sub <- subset(blackft.sub, blackft.sub$conditionNumber < 1.85 & 
+                         blackft.sub$conditionNumber > 0.65)
> 
\end{Sinput}
\end{Schunk}

\item Build a model for weight as a function of length. Discuss your model and the associated diagnostic plots.

\begin{Schunk}
\begin{Sinput}
> require(xtable)
> par(mfrow = c(2,2))
> fishlm <- lm(log(weight)~log(length)*species + log(length)*year + species*year + 
+                section*year, na.omit(blackft.sub))
> plot(fishlm)
\end{Sinput}
\end{Schunk}

